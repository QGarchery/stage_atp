\title{Stage de M2 MPRI: Démonstration automatique en Coq}

\section{Introduction}
Assistants de preuves 
Automatisation 
lien SAT

\newpage



\section{Intégration de solver SAT/SMT en Coq}

\subsection{Introduction}
\subsubsection{SAT/SMT}
SMTcoq fonctionne sur un sous ensemble de formules coq :
Ces formules sont en forme prénexe et sur un domaine décidable.

\subsection{fonctionnement de smtcoq}
La première étape est la réification. Il s'agit de transformer une formule coq en un AST:
ce n'est faisable que dans un autre langage que Coq, ici Ocaml.
Cet AST est ensuite envoyé à un prouveur smt, qui nous donne un certificat de preuve.
Il s'agit ensuite de rejouer ce certificat en coq. Il faut pour cela l'interprété: création
de tables qui permettent d'exprimer ce certificat, la formule, les sous-formules apparaissant
dans le certificat, les atomes, les lemmes utilisés. Il y a aussi une étape d'adaptation de ces
certificats: il peut manquer des étapes, la forme n'est pas la même que la notre ...

\subsection{fonctionnement d'un smt solver}
cnf : conjonction de clauses
clause : disjonction de litéraux
literal : atom ou negation d'atome
atome : ensemble (fini?) fixé
on peut voir les lemmes de la théorie comme des atomes. Ils sont utilisés par les prouveurs.
fonctionnement d'un smt prouveur :
-faire sat sur la formule en considérant les atomes comme indéfinis
-en déduire une instantiation qui marche, sinon renvoyer unsat
-envoyer cette instantiation aux prouveurs
-si tous les prouveurs renvoient vrai alors c'est sat
-sinon les prouveurs donnent un lemme additionnel
-rajouter ce lemme et recommencer (il y a un nombre fini d'instanciation des atomes)

\subsection{Certificats et small checker}
Un certificat est une liste de certif * position suivi d'une position finale.
Un certif est un type somme de scertif.
Un schecker prend un scertif et calcule une nouvelle formule (il faut lui doner l'état aussi).
Le checker sera simplement la combinaison par filtrage de ces schecker.
Pour vérifier que le certificat est correct, il faut appliquer dans l'ordre checker sur
chacun des certif et modifier la position correspondant à ce certif à chaque étape.
Il faut à la toute fin vérifier que la position finale dans l'état est la clause vide.

\subsection{Exemples de certificats et de vérificateurs}
\subsubsection{resolution chains}
C'est simplement une liste de clause.
Le schecker associé fait un fold dessus en appliquant la regle de resolution dessus. Cette regle est
refutationnellement complete.

\subsubsection{nor\_certif}
Il implémente la regle ~(A1 \/ A2 \/ ... \/ An) => ~Ai

\subsubsection{congruence theory}
Qui implémente les regles de transitivité et de congruence.



\section{veriT et zChaff}
Les certificats de zChaff:
-maintiennent un état des variables
-donnent un ordre d'execution (L : 2)
-définissent des clauses apprises au début du fichier par résolution
-définissent des commandes fixant la valeur d'une variable (Var : 5) à vrai (V : 1) ou faux (V : 0)
-définissent une clause de conflit (qui explicite les litéraux de conflit)

Il y a différentes stratégies pour l'implémentation de dpll, on peut choisir de donner une priorité
plus importante à la théorie ou aux solveur SAT, à plusieurs niveaux, priorité des regles de Basic dpll. 

La résolution de deux clauses utilise une dérivée de la fonction merge de merge sort.

La preuve par réflection utilise le principe d'égalité calculatoire de Coq: deux termes sont
toujours considérés à beta-red près. Attention : les définitions inductives ne définissent pas
de regles calculatoires additionnelles alors que Definition et Fixpoint si.


\section{preprocessing du certificat de veriT pour forall_inst}

Afin de ne pas avoir à rajouter des binders dans les proof witness de veriT, nous avons choisi de faire un
preprocessing dans les certificats renvoyés par veriT.
Les règles instance apparaissent sous la forme suivante :
     (forall_inst (or (not lemma) lemma_inst) id)
où lemma est un des lemmes rajoutés par l'utilisateur et lemma_inst est une instance de ce même lemme.

Pour chacune de ces règles instances, on garde une trace du lemme auquel elle se réfère puis on modifie la valeur
de cette règle: ce n'est plus (or (not lemma) lemma_inst) mais seulement (lemma_inst). En contrepartie, les règles
de résolution qui utilisent à la fois une de ces règles instances et le lemme associé sont réduites à seulement
l'instance du lemme.

Exemple :
si on a ensuite
(resolution (thm) id1 id2 id3)
et que id2 donne la règle instance écrite ci-dessus et que id3 donne la règle
(input (lemma))
alors on remplace cette règle résolution par une règle de résolution ayant id1 et id2.

On a fait également attention au fait que les règles temporaires de veriT donnent des alias des lemmes de départ
et qu'après une règle forall_inst il y a une règle qui transforme le OR interne en OR externe.

\section




\end{document}
