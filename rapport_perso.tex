\title{Stage de M2 MPRI: Démonstration automatique en Coq}

\section{Introduction}
Assistants de preuves 
Automatisation 
lien SAT

\newpage



\section{Intégration de solver SAT/SMT en Coq}

\subsection{Introduction}
\subsubsection{SAT/SMT}
SMTcoq fonctionne sur un sous ensemble de formules coq :
Ces formules sont en forme prénexe et sur un domaine décidable.

\subsection{fonctionnement de smtcoq}
La première étape est la réification. Il s'agit de transformer une formule coq en un AST:
ce n'est faisable que dans un autre langage que Coq, ici Ocaml.
Cet AST est ensuite envoyé à un prouveur smt, qui nous donne un certificat de preuve.
Il s'agit ensuite de rejouer ce certificat en coq. Il faut pour cela l'interprété: création
de tables qui permettent d'exprimer ce certificat, la formule, les sous-formules apparaissant
dans le certificat, les atomes, les lemmes utilisés. Il y a aussi une étape d'adaptation de ces
certificats: il peut manquer des étapes, la forme n'est pas la même que la notre ...

\subsection{fonctionnement d'un smt solver}
cnf : conjonction de clauses
clause : disjonction de litéraux
literal : atom ou negation d'atome
atome : ensemble (fini?) fixé
on peut voir les lemmes de la théorie comme des atomes. Ils sont utilisés par les prouveurs.
fonctionnement d'un smt prouveur :
-faire sat sur la formule en considérant les atomes comme indéfinis
-en déduire une instantiation qui marche, sinon renvoyer unsat
-envoyer cette instantiation aux prouveurs
-si tous les prouveurs renvoient vrai alors c'est sat
-sinon les prouveurs donnent un lemme additionnel
-rajouter ce lemme et recommencer (il y a un nombre fini d'instanciation des atomes)

\subsection{Certificats et small checker}
Un certificat est une liste de certif * position suivi d'une position finale.
Un certif est un type somme de scertif.
Un schecker prend un scertif et calcule une nouvelle formule (il faut lui doner l'état aussi).
Le checker sera simplement la combinaison par filtrage de ces schecker.
Pour vérifier que le certificat est correct, il faut appliquer dans l'ordre checker sur
chacun des certif et modifier la position correspondant à ce certif à chaque étape.
Il faut à la toute fin vérifier que la position finale dans l'état est la clause vide.

\subsection{Exemples de certificats et de vérificateurs}
\subsubsection{resolution chains}




\subsubsection{nor\_certif}
\subsubsection{congruence theory}




\end{document}
